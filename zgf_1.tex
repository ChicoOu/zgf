\documentclass[11pt,a4paper,UTF8]{ctexbook}
\usepackage{amsmath,mathrsfs,color}
\ctexset{
	part/pagestyle=empty,
	chapter={
		name={第,章},
		number=\arabic{chapter},
	},
  section = {
    format = \raggedright\Large\bfseries,
  }
}

\numberwithin{equation}{chapter}
\title{曾国藩全集}
\author{欧志球}
\date{2019年3月}
\begin{document}
	\maketitle
  \chapter{曾国藩年谱}
  \section{嘉庆十六年至咸丰二年}
  \emph{1811年},嘉庆十六年(\textcolor{blue}{辛未年})十月十一日亥时
  (\textcolor{blue}{21点到23点}),\textbf{出生} \\
   \emph{1824年},道光四年,十四岁,与\textcolor{blue}{欧阳沧溟女儿订婚},与父亲
   赴长沙 \\
   \emph{1826年},道光六年,十六岁,长沙府试,第七名 \\
   \emph{1833年},道光十二年,二十二岁,佾生\footnote{yi}(\textcolor{blue}{成绩优秀备取})注册 \\
   \emph{1834年},道光十三年,二十三岁,与欧阳夫人\textbf{结婚} \\
   \emph{1836年},道光十五年,二十五岁,北京长沙会馆,会试没成功 \\
   \emph{1837年},道光十六年,二十六岁,会试仍没成功 \\
   \emph{1838年},道光十七年,二十七岁,在长沙与\textbf{刘蓉}以及\textbf{郭嵩涛}
   在长沙应试。十月生\textcolor{blue}{桢第} \\
   \emph{1839年},\textcolor{red}{道光十八年,二十八岁,中进士。\textbf{宁乡梅钟
       澎,茶陵陈源衮}为同年}。授翰林院\textcolor{green}{庶吉士} \\
   \emph{1840年},第一个儿子\textcolor{blue}{桢第}以及最小的妹妹染天花去世。弟弟
   \textcolor{blue}{国华}过继给叔父\textcolor{blue}{高轩公}。生第二个儿子
   \textcolor{blue}{纪泽} \\
   \emph{1841年},道光二十年,三十岁,大病\footnote{六月至九月},
   \textcolor{blue}{欧阳兆熊}与\textcolor{blue}{吴廷栋}救治。十二月,父亲、夫人、
   弟弟国荃以及儿子纪泽到北京 \\
   \emph{1843年},道光二十二年,三十二岁,潜心程朱理学,与\textcolor{blue}{倭仁、
     吴廷栋、何桂珍、仁和、邵懿辰、陈源衮}等讨论。 \\
   \emph{1844年},道光二十三年,三十三岁,\textcolor{red}{充四川正考官,
     再次大病,\textbf{该差事得俸禄千金!}} \\
   \emph{1845年},道光二十四年,三十四岁,再任庶吉士。因\textcolor{blue}{郭嵩涛}
   遇\textcolor{blue}{江忠源},指其“必立功名于天下,然当以节义死”。 \\
\end{document}