\documentclass[11pt,a4paper,UTF8]{ctexbook}
\usepackage{amsmath,mathrsfs,color}
\ctexset{
	part/pagestyle=empty,
	chapter={
		name={第,章},
		number=\arabic{chapter},
	},
  section = {
    format = \raggedright\Large\bfseries,
  }
}

\numberwithin{equation}{chapter}
\title{曾国藩全集}
\author{欧志球}
\date{2019年3月}
\begin{document}
	\maketitle
  \chapter{曾国藩年谱}
  \section{嘉庆十六年至咸丰二年}
  \emph{1811年},嘉庆十六年(\textcolor{blue}{辛未年})十月十一日亥时
  (\textcolor{blue}{21点到23点}),\textbf{出生} \\
   \emph{1824年},道光四年,十四岁,与\textcolor{blue}{欧阳沧溟女儿订婚},与父亲
   赴长沙 \\
   \emph{1826年},道光六年,十六岁,长沙府试,第七名 \\
   \emph{1833年},道光十二年,二十二岁,佾生\footnote{yi}(\textcolor{blue}{成绩优秀备取})注册 \\
   \emph{1834年},道光十三年,二十三岁,与欧阳夫人\textbf{结婚} \\
   \emph{1836年},道光十五年,二十五岁,北京长沙会馆,会试没成功 \\
   \emph{1837年},道光十六年,二十六岁,会试仍没成功 \\
   \emph{1838年},道光十七年,二十七岁,在长沙与\textbf{刘蓉}以及\textbf{郭嵩涛}
   在长沙应试。十月生\textcolor{blue}{桢第} \\
   \emph{1839年},\textcolor{red}{道光十八年,二十八岁,中进士。\textbf{宁乡梅钟
       澎,茶陵陈源衮}为同年}。授翰林院\textcolor{green}{庶吉士} \\
   \emph{1840年},第一个儿子\textcolor{blue}{桢第}以及最小的妹妹染天花去世。弟弟
   \textcolor{blue}{国华}过继给叔父\textcolor{blue}{高轩公}。生第二个儿子
   \textcolor{blue}{纪泽} \\
   \emph{1841年},道光二十年,三十岁,大病\footnote{六月至九月},
   \textcolor{blue}{欧阳兆熊}与\textcolor{blue}{吴廷栋}救治。十二月,父亲、夫人、
   弟弟国荃以及儿子纪泽到北京 \\
   \emph{1843年},道光二十二年,三十二岁,潜心程朱理学,与\textcolor{blue}{倭仁、
     吴廷栋、何桂珍、仁和、邵懿辰、陈源衮}等讨论。 \\
   \emph{1844年},道光二十三年,三十三岁,\textcolor{red}{充四川正考官,
     再次大病,\textbf{该差事得俸禄千金!}} \\
   \emph{1845年},道光二十四年,三十四岁,再任庶吉士。因\textcolor{blue}{郭嵩涛}
   遇\textcolor{blue}{江忠源},指其“必立功名于天下,然当以节义死”。 \\
   \emph{1847年},道光二十六年,三十六岁,与国潢、国华砥砺于学。“\emph{近世为学者,
     不以身心切近为务,恒视一时之风尚以为程而趋之。不数年风尚稍变,又弃其所业以
     趋于新...}” \\
   \emph{1848年},道光二十七年,三十七岁,六月奉旨\textcolor{red}{升授内阁学士,
     兼吏部侍郎}\footnote{正式发迹的开始} \\
   \emph{1849年},道光二十八年,三十八岁,制《曾氏家训长篇》,补
   \textcolor{blue}{秦文恭}的《五礼通考》;条分近代学术\footnote{精力过人,心力
     过人} \\
   \emph{1850年},道光二十九年,三十九岁,授吏部右侍郎。重点:
   \begin{itemize}
     \item{有事加班不待期日}
     \item{取孔子木主焚化,而为文以祀韩子\footnote{有见识、也有胆识}}
     \item{兼署兵部右侍郎、参与各类阅卷大臣。\textcolor{red}{这么多事情,遇上不
           可拒绝的老板,如何处理过来的?}}
   \end{itemize} 
   \emph{1851年},道光三十年,四十岁,\textcolor{blue}{道光}驾崩,曾继续得到\textcolor{blue}{咸丰}
   的赏识。 任工部左侍郎、兵部左侍郎以及各类阅卷大臣。本年度洪、杨太平天国起,是
   为曾进一步发展之机遇;\textcolor{blue}{林则徐}病死在赴任路上,否则未必太平天
   国能崛起;\\
   \emph{1852年},咸丰元年,四十一岁。奏敬陈圣德三端,预防流弊一折,
   \textcolor{red}{多切直之言},但咸丰嘉纳。 \\
   \emph{1853年},咸丰二年,四十二岁。\textcolor{red}{母亲江氏去世,守制丁忧在家}。
   太平天国入湖南,影响很大,咸丰下旨,郭嵩涛等劝曾国藩出来。几件事情很重要:
   \begin{itemize}
     \item[*]{招募乡勇}
     \item[*]{仿戚继光,操练,并制定《训练章程》}
   \end{itemize}
  \section{咸丰三年}
  \emph{1854年},咸丰三年,四十三岁。

   “现在安(徽)省待援甚急,若必偏执己见,则太觉迟缓。朕知\emph{汝尚能激发天良},
   故特命汝赴援,以济燃眉。今观汝奏,直以数省军务一身克当,\emph{试问汝之才力能
     乎?否乎?平时漫自矜诩,以为无出己之右者。}及至临事果能尽符其言甚好。若
   \emph{稍涉张皇,岂不贻笑于天下?}着设法赶紧赴援,能早一步即得一步之益
   $\ldots$” \\

   【曾国藩的应对】十六日得到咸丰的朱批,二十一日\emph{具疏逐条陈明},除说明各种
   原因以外,最后一条\emph{“饷乏兵单,成效不敢必,唯有愚诚不敢避死而已。与其将来
     毫无功绩,受大言欺君之罪,不如此时据实陈明,受畏葸不前之罪。”}

  \section{咸丰四年}
  \emph{1855年},咸丰四年,四十四岁。
\end{document}